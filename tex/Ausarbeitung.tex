\documentclass[conference,compsoc,final,a4paper, onecolumn, 11pt]{IEEEtran}
\usepackage[utf8]{inputenx}

\newcommand{\autoren}[0]{Shiner, Nicole \and Keller, Joshua \and Rühm, Moritz}
\newcommand{\dokumententitel}[0]{Urbanisierung als Trend und Beschleuniger der digitalen Transformation}

\usepackage[pdftex]{graphicx}
\graphicspath{{img/}}
\DeclareGraphicsExtensions{.pdf,.jpeg,.jpg,.png}
\usepackage[cmex10]{amsmath}
\usepackage{algorithmic}
\usepackage{array}
\usepackage{dblfloatfix}
\usepackage{url}
\usepackage[autostyle=true,german=quotes]{csquotes}
\usepackage[backend=biber,
            sorting=none,   % Keine Sortierung
            doi=true,       % DOI anzeigen
            isbn=false,     % ISBN nicht anzeigen
            url=true,       % URLs anzeigen
            maxnames=6,     % Ab 6 Autoren et al. verwenden
            minnames=1,     % und nur den ersten Autor angeben
            style=ieee,]{biblatex}
\usepackage{booktabs}
\usepackage{xcolor}
\usepackage{listings} % Source Code listings
\usepackage[printonlyused]{acronym}
\usepackage{fancyvrb}
\usepackage{tocloft} % Schönere Inhaltsverzeichnisse

% Farben definieren
\definecolor{linkblue}{RGB}{0, 0, 100}
\definecolor{linkblack}{RGB}{0, 0, 0}
\definecolor{darkgreen}{RGB}{14, 144, 102}
\definecolor{darkblue}{RGB}{0,0,168}
\definecolor{darkred}{RGB}{128,0,0}
\definecolor{comment}{RGB}{63, 127, 95}
\definecolor{javadoccomment}{RGB}{63, 95, 191}
\definecolor{keyword}{RGB}{108, 0, 67}
\definecolor{type}{RGB}{0, 0, 0}
\definecolor{method}{RGB}{0, 0, 0}
\definecolor{variable}{RGB}{0, 0, 0}
\definecolor{literal}{RGB}{31,0, 255}
\definecolor{operator}{RGB}{0, 0, 0}

\usepackage[ngerman]{betababel}

% Immer et al. sagen, auch bei Deutsch als Sprache
\DefineBibliographyStrings{ngerman}{
    andothers = {{et al\adddot}},
}
\usepackage[
      unicode=true,
      hypertexnames=false,
      colorlinks=true,
      colorlinks=false,
      linkcolor=darkblue,
      citecolor=darkblue,
      urlcolor=darkblue,
      pdftex
   ]{hyperref}
%	 \PrerenderUnicode{ü}


% Einstellungen für Quelltexte
\lstset{
    xleftmargin=0.1cm,
    basicstyle=\scriptsize\ttfamily,
    keywordstyle=\color{keyword},
    identifierstyle=\color{variable},
    commentstyle=\color{comment},
    stringstyle=\color{literal},
    tabsize=2,
    lineskip={2pt},
    columns=flexible,
    inputencoding=utf8,
    captionpos=b,
    breakautoindent=true,
    breakindent=2em,
    breaklines=true,
    prebreak=,
    postbreak=,
    numbers=none,
    numberstyle=\tiny,
    showspaces=false,      % Keine Leerzeichensymbole
    showtabs=false,        % Keine Tabsymbole
    showstringspaces=false,% Leerzeichen in Strings
    morecomment=[s][\color{javadoccomment}]{/**}{*/},
    literate={Ö}{{\"O}}1 {Ä}{{\"A}}1 {Ü}{{\"U}}1 {ß}{{\ss}}2 {ü}{{\"u}}1 {ä}{{\"a}}1 {ö}{{\"o}}1
}

\hypersetup{
    pdftitle={\dokumententitel},
    pdfauthor={\autoren},
    pdfdisplaydoctitle=true,
    hidelinks
}

% Makros für typographisch korrekte Abkürzungen
\newcommand{\zb}[0]{z.\,B.}
\newcommand{\dahe}[0]{d.\,h.}
\newcommand{\ua}[0]{u.\,a.}

% Sourcecode
\newcommand{\srcloc}{src/}

% Literatur einbinden
\addbibresource{citations.bib}


\begin{document}

% Titel
\title{\dokumententitel}

% Autoren
\author{
  \IEEEauthorblockN{Shiner, Nicole}
  \and
  \IEEEauthorblockN{Keller, Joshua}
  \IEEEauthorblockA{
    \\
    Hochschule Mannheim\\
    Fakultät für Informatik\\
    Paul-Wittsack-Str. 10,
    68163 Mannheim
  }
  \and
  \IEEEauthorblockN{Rühm, Moritz}
}

% Titel erzeugen
\maketitle
\thispagestyle{plain}
\pagestyle{plain}

% Dokument
% ----------------------------------------------------------------------------------------------------------
% Abstract
\begin{abstract}
Hier Abstract.
\end{abstract}

% Inhaltsverzeichnis
{\tableofcontents}


% Einleitug
% -------------------------------------------------------
\section{Einleitung}
Hier Einleitung.


% Grundlagen und Kontext
% -------------------------------------------------------
\section{Grundlagen und Kontext}
\subsection{Digitale Transformation}
Lorem Ipsum

\subsection{Urbanisierung}
Lorem Ipsum

\subsection{Verbindung oder sowas}
Lorem Ipsum


% Wechselwirkungen zwischen Urbanisierung und digitaler Transformation
% -------------------------------------------------------
\section{Wechselwirkungen zwischen Urbanisierung und digitaler Transformation}
\subsection{Warum Urbanisierung die digitale Transformation beschleunigt}
Lorem Ipsum

\subsection{Urbane Herausforderungen, die durch digitale Transformation bewältigt werden können}
Lorem Ipsum

\subsection{Grenzen, Risiken und ethische Aspekte der digitalen Stadtentwicklung}
Lorem Ipsum


% Digitale Transformation als Lösung für urbane Herausforderungen
% -------------------------------------------------------
\section{Digitale Transformation als Lösung für urbane Herausforderungen}
\subsection{Fallstudien und Praxisbeispiele}
Lorem Ipsum

\subsection{Zukunftsperspektiven: Die Stadt der Zukunft}
Lorem Ipsum


% Cheatsheet (Hilfskapitel)
% -------------------------------------------------------
\section{Cheatsheet}
\subsection{Textformatierungen}
\begin{itemize}
  \item \texttt{\textbackslash textbf\{fett\}}: \textbf{Fettgedruckter Text.}
  \item \texttt{\textbackslash textit\{kursiv\}}: \textit{Kursiver Text.}
  \item \texttt{\textbackslash underline\{unterstrichen\}}: \underline{Unterstrichener Text.}
  \item \texttt{\textbackslash emph\{betont\}}: \emph{Betonter Text.}
\end{itemize}

\subsection{Zitate}
Zitate werden mit \verb|\autocite{key}| geschrieben. 
Der Key ist das was als erstes in den geschweiften klammern steht.\\
Bsp.: \verb|\autocite{urbach_digitalization_2021}| \\
Der Rest wird automatisch gemacht. \autocite{urbach_digitalization_2021}

\subsection{Abkürzungen}
Abkürzungen werden mit \verb|\ac{kürzel}| geschrieben und müssen vor benutzung einmal am unteren Ende des dokuments mit \verb|\acro{Kürzel}{Lange version}| beschrieben werden. \\
Bsp.: \verb|\ac{dT}| \text{→} im Text \\\\
\verb|\begin{acronym}[IEEE]| \\
\verb|\acro{dT}{digitale Transformation}| \text{→} Unten im Dokument\\
\verb|\end{acronym}| \\\\
So wird die Abkürzung bei der ersten benutzung ausgeschrieben und dannach als richtige Abkürzung benutzt. \\
\ac{dT} \\
\ac{dT}

\subsection{Listen}
\begin{itemize}
    \item Unnummerierte Liste:
    \begin{verbatim}
    \begin{itemize}
        \item Erster Punkt
        \item Zweiter Punkt
    \end{itemize}
    \end{verbatim}
    \item Nummerierte Liste:
    \begin{verbatim}
    \begin{enumerate}
        \item Erster Punkt
        \item Zweiter Punkt
    \end{enumerate}
    \end{verbatim}
\end{itemize}

\subsection{Bilder einfügen}
\texttt{\textbackslash includegraphics\{bild.png\}} \\
Bild in den image Ordner packen.

\subsection{Links}
\begin{itemize}
    \item Link zum clicken mit custom Text: \texttt{\textbackslash href\{URL\}\{Text\}}.
    \item Direkte URL: \texttt{\textbackslash url\{URL\}}.
\end{itemize}


% --------------------------------------------------------------------
\section*{Abkürzungen}
\addcontentsline{toc}{section}{Abkürzungen}

% Die längste Abkürzung wird in die eckigen Klammern
% bei \begin{acronym} geschrieben, um einen hässlichen
% Umbruch zu verhindern
\begin{acronym}[IEEE]
\acro{dT}{digitale Transformation}
\end{acronym}

% Literaturverzeichnis
\addcontentsline{toc}{section}{Literatur}
\printbibliography

\end{document}
